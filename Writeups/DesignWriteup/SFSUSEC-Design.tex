%%% PRACE GENERIC LAYOUT; DO NOT CHANGE %%%
\documentclass{prace}
%%% END OF PRACE GENERIC LAYOUT %%%

% TITLE
%   - use the name of your project
%   - capitalise the first letter
\title{Initial Design Document}
\date{31.01.2024}

% Instead of a DOI, we will use the name of the club
\doi{CyberSecurity Club at SFSU}

% AUTHORS
\author[1]{Ethan Hanlon}
\author[1]{Michael Petrossian}

% AFFILIATIONS
\affiliation{San Francisco State University, 1600 Holloway Avenue, San Francisco, CA 94132, USA}

%%% PRACE GENERIC LAYOUT; DO NOT CHANGE %%%
\begin{document}
\maketitle
%%% END OF PRACE GENERIC LAYOUT %%%

% ABSTRACT
%   - write a concise abstract that outlines the approach / methods, main 
%     results, and relevance of your project
\begin{abstract}
This research focuses on designing and implementing a secure MISC device that will be built to adhere to the standards of the functional and security requirements. Our design will 
aim to withstand all attack scenarios presented by the MITRE documentation, as well as provide a new approach to embedded system programming with the popular and recent memory-safe Rust programming language. 
\end{abstract}

\section{Introduction}

Lorem ipsum dolor sit amet, consectetuer adipiscing elit. Sed posuere interdum
sem. Quisque ligula eros ullamcorper quis, lacinia quis facilisis sed sapien.
Mauris varius diam vitae arcu. Sed arcu lectus auctor vitae, consectetuer et
venenatis eget velit. Sed augue orci, lacinia eu tincidunt et eleifend nec
lacus. Donec ultricies nisl ut felis, suspendisse potenti. Lorem ipsum ligula
ut hendrerit mollis, ipsum erat vehicula risus, eu suscipit sem libero nec
erat. Aliquam erat volutpat. Sed congue augue vitae neque. Nulla consectetuer
porttitor pede. Fusce purus morbi tortor magna condimentum vel, placerat id
blandit sit amet tortor.


\section{Functional Requirements}

We plan to meet the functional requirements by doing x, y, and z.

\section{Security Requirements}

\subsection{Security Requirement 1}
Security Requirement 1 (\textbf{SR1}) requires us to ensure the Application Processor will not boot unless
expected components are present and valid. Our plan for this involves a two-pronged approach
which uses public/private key encryption and a secure cryptographic nonce.

During the build process, we will generate a public/private key pair. The public key will be
flashed onto the components, and the private key will be stored in the application processor.
Additionally, a random seed will be generated using /dev/urandom and stored in the application
processor. This seed will be used by the components and the application processor to generate
and validate cryptographic nonces.

The components will use the public key to encrypt the nonce along with their own
unique identifier and send it back to the application processor over I2C bus. The application
processor will use the private key to decrypt the nonce and verify that it matches the original
nonce. If the nonces match, the application processor will know that the components are valid
and will continue with boot. If no message is received, or the nonces do not match, the application
processor will not boot.

\subsection{Security Requirement 2}
Security Requirement 2 (\textbf{SR2}) requires us to ensure that the components will not boot until commanded
to do so by an Application Processor that has validated the integrity of the system. Our plan to
meet this requirement is much the same as the plan for Security Requirement 1, but in reverse.

During the build process, a random seed will be burned onto the firmware as delineated in 
SR1. In addition, each component will receive a public/private key pair similar to the AP's.
This key will be used to encrypt messages before transmitting them over the I2C bus, as well
as to verify digital signatures.

After the application processor verifies the message sent by the components as delineated
in SR1, it will return an acknowledgement, encrypted with the component's public key. Upon
receipt of this acknowledgement, the component will use its private key to decrypt the 
message and validate its digital signature. From there, the nonce will be checked against
the random key generator. If the message is not present, the digital signature is invalid,
or the nonce verification fails, the component will immediately halt the boot process.
Conversely, if all checks pass - suggesting the AP successfully validated the components -
the component will proceed with the boot sequence.

% REFERENCE LIST
%   - use \thebibliography and \bibitems to enter references, no separate .bib
%     files
%   - use normal font for *everything* (no bold typefaces etc.)
%   - shorten the last page number, i.e. 51--9 for pages 51--59
%   - for more than 6 authors the first 6 should be listed followed by et al.
%     - use \emph{et al.} for the et al.
%   - end each reference with a period
%
% example:
%   \begin{thebibliography}{99}
%     \bibitem{scholes-DiscussFaradaySoc-70}
%       S. Scholes, Discuss. Faraday Soc. No. 50 (1970) 222.
%     \bibitem{mazurin-Phase-Separation-in-Glass-84}
%       O.V. Mazurin and E.A. Porai-Koshits (eds.), 
%       Phase Separation in Glass, North-Holland, Amsterdam, 1984.
%     \bibitem{dimitriev-JMaterSci-75}
%       Y. Dimitriev and E. Kashchieva, J.Mater. Sci. 10 (1975) 1419.
%     \bibitem{eaton-Porous-Glass-Support-Material-75}
%       D.L. Eaton, Porous Glass Support Material, US Patent No. 3 904 422 
%       (1975).
%   \end{thebibliography}
\begin{thebibliography}{99}
	\bibitem{scholes-DiscussFaradaySoc-70}
	S. Scholes, Discuss. Faraday Soc. No. 50 (1970) 222.
	\bibitem{someone-SomeJournal-00}
	O.V. Mazurin and E.A. Porai-Koshits (eds.),
\end{thebibliography}

% ACKNOWLEDGEMENTS
%   - additional acknowledgements may be added
%   - names of PRACE machines and the corresponding sites and countries
%     should be inserted to end of the general PRACE acknowledgement 
\subsection*{Acknowledgements}
This template was originally created by the Partnership for Advanced Computing
in Europe (PRACE). It was modified for use by the San Francisco State University
eCTF team.

Make sure to include acknowledgements for people who helped in the project,
including professors, graduate students, and other team members.

%%% PRACE GENERIC LAYOUT; DO NOT CHANGE %%%
\end{document}
%%% END OF PRACE GENERIC LAYOUT %%%
