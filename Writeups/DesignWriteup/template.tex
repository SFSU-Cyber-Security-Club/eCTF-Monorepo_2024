%%% Partnership for Advanced Computing in Europe 
%%%   www.prace-ri.eu
%%%
%%% LaTeX template for a PRACE white paper.
%%%
%%% (c) CSC - IT Center for Science Ltd.
%%%     author: Martti Louhivuori (martti.louhivuori@csc.fi)
%%%
%%% Generic instructions:
%%%   - follow the point-by-point instructions
%%%   - fill in the required author information, title, and abstract
%%%   - write your paper using the general format outlined below
%%%   - do NOT touch the generic layout between the following tags: 
%%%       %%% PRACE GENERIC LAYOUT; DO NOT CHANGE %%%
%%%       %%% END OF PRACE GENERIC LAYOUT %%%
%%%   - check any guidelines for the expected length of the paper
%%%   - refer to 'example.tex' and 'example.pdf' for a practical example
%%%
%%% PRACE GENERIC LAYOUT; DO NOT CHANGE %%%
\documentclass{prace}
%%% END OF PRACE GENERIC LAYOUT %%%

% TITLE
%   - use the name of your project
%   - capitalise the first letter
\title{Initial Design Document}
\date{31.01.2024}

% Please inquire about a reserved DOI for the white paper. Then add
% the DOI prior to producing a final version
\doi{CyberSecurity Club at SFSU}

% AUTHORS
%   - include all people involved in the effort
%     - depending on their contribution, include PRACE experts as authors 
%       or mention them in acknowledgements
%   - give affiliations in the option field as a list of numbers 
%     corresponding to the order of \affiliation definitions, i.e. 
%     [1] -> 1st \affiliation, [2] -> 2nd, [1,2] -> 1st & 2nd
%   - mark one of the authors as the corresponding author using
%     \corresponding before the \author, i.e. 
%       \corresponding\author[1]{N.N.}
%
% example:
%   \author[1]{First Author}
%   \corresponding\author[2]{Second Author}
%   \author[1,2]{Third Author}
\author[1]{Ethan Hanlon}
\author[1]{Michael Petrossian}

% AFFILIATIONS
%   - define affiliations in the same order you used for in the author 
%     definitions
%   - include: name, address, city, postcode, and country
%
% example:
%   \affiliation{First affiliation, Address, City and Postcode, Country}
%   \affiliation{Second affiliation, Address, City and Postcode, Country}
\affiliation{San Francisco State University, 1600 Holloway Avenue, San Francisco, CA 94132, USA}

% CONTACT INFORMATION
%   - give the email address of the corresponding author
%
% example:
%   \email{second.author@example.com}
\email{}

%%% PRACE GENERIC LAYOUT; DO NOT CHANGE %%%
\begin{document}
\maketitle
%%% END OF PRACE GENERIC LAYOUT %%%

% ABSTRACT
%   - write a concise abstract that outlines the approach / methods, main 
%     results, and relevance of your project
\begin{abstract}
Lorem ipsum dolor sit amet, consectetur adipiscing elit. Nulla euismod auctor purus,
sit amet aliquam arcu pellentesque non. Fusce a quam dui. Nullam porta venenatis ligula
maximus pharetra. Nulla diam nulla, mattis nec imperdiet nec, convallis a lacus. Sed 
diam libero, hendrerit ac condimentum a, laoreet eu est. Phasellus varius augue id nisl
sagittis fringilla. Nam elementum aliquet ultrices. Donec rhoncus tristique felis id
suscipit. Etiam sit amet risus eu ipsum lacinia condimentum sed ac felis. Suspendisse
aliquam vel turpis sed ultrices. Proin orci dolor, varius ut faucibus et, placerat
pulvinar orci. Donec vitae magna non nunc imperdiet efficitur sed ac elit.
\end{abstract}

% MAIN BODY
%
% Consider write the report in the style of a journal article (i.e. Introduction, 
% Methods, Results, Conclusions). The appropriate length of the paper depends on the 
% white paper in question (it may vary for SHAPE projects or more technical reports)
%
% In the report please describe:
%   - goals of the project
%     - scientific case and goals related to the project
%     - technical goals (performance, parallel scalability, ...)
%   - work done in the project, including
%     - technical and algorithmic methods and programming techniques employed
%     - use of profiling tools when applicable
%     - use of numerical libraries when applicable
%     - machine(s) used for the work
%   - results obtained
%     - give quantitative measurements of the achieved performance 
%       enhancements and the scaling behaviour
%     - discuss how the results compare with the goals
%   - summary
%     - relevance of the obtained results for the stated scientific goals
%     - outlook on possible future work
%
% instructions:
%   - use only \sections and \subsections to divide the paper into logical 
%     segments
%   - capitalise only the first letter of headings
%   - symbols denoting vectors and matrices should be in bold type
%   - scalar variables should be in italics, i.e. enclosed within $$ in text
%   - weights and measures should be expressed in SI units, mind the non-breaking 
%     space between number+unit, preferrably use siunitx package 
%   - avoid footnotes if at all possible
%   - collate acknowledgements in a separate section at the end of the 
%     article; do NOT include them on the title page, as a footnote etc.
%
% example:
%   \section{Introduction}
%     Introductory text...
%   \section{Methods}
%     General description...
%   \subsection{Specific method A}
%     Method A in detail...
%   ...
%   \section{Acknowledgements}
%     The results in this paper have been achieved using the PRACE Research
%     Infrastructure. 
%     
\section{Introduction}

Lorem ipsum dolor sit amet, consectetuer adipiscing elit. Sed posuere interdum
sem. Quisque ligula eros ullamcorper quis, lacinia quis facilisis sed sapien.
Mauris varius diam vitae arcu. Sed arcu lectus auctor vitae, consectetuer et
venenatis eget velit. Sed augue orci, lacinia eu tincidunt et eleifend nec
lacus. Donec ultricies nisl ut felis, suspendisse potenti. Lorem ipsum ligula
ut hendrerit mollis, ipsum erat vehicula risus, eu suscipit sem libero nec
erat. Aliquam erat volutpat. Sed congue augue vitae neque. Nulla consectetuer
porttitor pede. Fusce purus morbi tortor magna condimentum vel, placerat id
blandit sit amet tortor.

\subsection{Methods}

Mauris sed libero. Suspendisse facilisis nulla in lacinia laoreet, lorem velit
accumsan velit vel mattis libero nisl et sem. Proin interdum maecenas massa
turpis sagittis in, interdum non lobortis vitae massa. Quisque purus lectus,
posuere eget imperdiet nec sodales id arcu. Vestibulum elit pede dictum eu,
viverra non tincidunt eu ligula.

% FIGURES
%   - all photographs, schemas, graphs, and diagrams are to be referred to 
%     as figures
%   - line drawings should be good quality scans or true electronic ouput, 
%     perferrably vector format
%   - low-quality scans are not acceptable
%   - lettering and symbols should be clearly defined either in the caption 
%     or in a legend provided as part of the figure
%   - figures should be placed at the top or the bottom of a page whenever
%     possible
%   - if two images fit next to each other, they may be placed so to save
%     space
%   - refer to figures in the text as Fig.~\ref{fig: foo} (with the correct 
%     reference labels of course)
%
% examples:
%   - single image:
%     \begin{figure}
%       \includegraphics[width=0.4\textwidth]{example}\hfill{}
%       \caption{singlepicture}
%       \label{fig: single-example}
%     \end{figure}
%
%   - two images side-by-side:
%     \begin{figure}
%       \includegraphics[width=0.4\textwidth]{example}\hfill{}
%       \includegraphics[width=0.4\textwidth]{example}\hfill{}
%       \caption{(a) first picture; (b) second picture}
%       \label{fig: double-example}
%     \end{figure}
% \begin{figure}
% 	\includegraphics[]{}\hfill{}
% 	\caption{}
% 	\label{}
% \end{figure}

% TABLES
%   - tables must be embedded into the text and not supplied separately
%   - caption before tabular
%   - left-justified columns
%   - only horisontal lines within a table:
%       - \toprule at the beginning of the table
%       - \midrule after the column headings but before the body
%       - \bottomrule at the end of the table
%   - \cmidrule can also be used for partial lines within the column headings 
%     (see the booktabs package documentation for further details: 
%      http://www.ctan.org/tex-archive/macros/latex/contrib/booktabs/ )
%   - refer to tables in the text as Table~\ref{tab: foo} (with the correct 
%     reference labels of course)
%
% example:
%   \begin{table}
%     \caption{An example of a table}
%     \label{tab: example}
%     \begin{tabular}{lll}
%       \toprule 
%       An example of a column heading & Column A (t) & Column B (T) \\
%       \midrule
%       An entry & 1 & 2 \\
%       and another entry & 3 & 4 \\
%       yet another entry & 5 & 6 \\
%       \bottomrule 
%     \end{tabular}
%   \end{table}

\begin{table}[h] % h = here, prevents table from floating to the top of the document
	\caption{A cool table}
	\label{}
	\begin{tabular}{lll}
		\toprule
		Column A & Column B & Column C \\
		\midrule
		1 & 2 & 3 \\
		\bottomrule
	\end{tabular}
\end{table}


% LISTS
%   - use \itemize for bulleted lists and \enumerate for numbered lists
%
% example:
%   \begin{itemize}
%     \item First point
%     \item Second point
%   \end{itemize}
\begin{itemize}
	\item Lorem
	\item Ipsum
\end{itemize}

% REFERENCES
%   - use \cite for references
%   - always include \cite even when referring to the authors by name
%
% example:
%   Example citation to a paper\cite{scholes-DiscussFaradaySoc-70} and to 
%   another paper by Someone \emph{et al.}\cite{someone-SomeJournal-00}.
% \cite{scholes-DiscussFaradaySoc-70}

This text contains a citation to a paper\cite{scholes-DiscussFaradaySoc-70} and to another paper by Someone \emph{et al.}\cite{someone-SomeJournal-00}.

% EQUATIONS
%   - use the equation environment for all equations
%   - short in-line notation may also be used, but should be avoided if
%     possible
%   - use bold type face (\mathbf) for vectors and matrices
%   - refer to equations in the text as Eq.~\ref{eq: foo} (with the correct 
%     reference labels of course)
%
% example:
%   \begin{equation}
%     Rt = K EP = 93.02 (\pm 9.62) – 13.45
%     \label{eq: example}
%   \end{equation}

Now look at this cool equation:

\begin{equation}
	Rt = K EP = 93.02 (\pm 9.62) - 13.45
	\label{eq: example}
\end{equation}

\section{Functional Requirements}

We plan to meet the functional requirements by doing x, y, and z.

\section{Security Requirements}

We plan to meet the security requirements by doing x, y, and z.

% REFERENCE LIST
%   - use \thebibliography and \bibitems to enter references, no separate .bib
%     files
%   - use normal font for *everything* (no bold typefaces etc.)
%   - shorten the last page number, i.e. 51--9 for pages 51--59
%   - for more than 6 authors the first 6 should be listed followed by et al.
%     - use \emph{et al.} for the et al.
%   - end each reference with a period
%
% example:
%   \begin{thebibliography}{99}
%     \bibitem{scholes-DiscussFaradaySoc-70}
%       S. Scholes, Discuss. Faraday Soc. No. 50 (1970) 222.
%     \bibitem{mazurin-Phase-Separation-in-Glass-84}
%       O.V. Mazurin and E.A. Porai-Koshits (eds.), 
%       Phase Separation in Glass, North-Holland, Amsterdam, 1984.
%     \bibitem{dimitriev-JMaterSci-75}
%       Y. Dimitriev and E. Kashchieva, J.Mater. Sci. 10 (1975) 1419.
%     \bibitem{eaton-Porous-Glass-Support-Material-75}
%       D.L. Eaton, Porous Glass Support Material, US Patent No. 3 904 422 
%       (1975).
%   \end{thebibliography}
\begin{thebibliography}{99}
	\bibitem{scholes-DiscussFaradaySoc-70}
	S. Scholes, Discuss. Faraday Soc. No. 50 (1970) 222.
	\bibitem{someone-SomeJournal-00}
	O.V. Mazurin and E.A. Porai-Koshits (eds.),
\end{thebibliography}

% ACKNOWLEDGEMENTS
%   - additional acknowledgements may be added
%   - names of PRACE machines and the corresponding sites and countries
%     should be inserted to end of the general PRACE acknowledgement 
\subsection*{Acknowledgements}
This template was originally created by the Partnership for Advanced Computing
in Europe (PRACE). It was modified for use by the San Francisco State University
eCTF team.

Make sure to include acknowledgements for people who helped in the project,
including professors, graduate students, and other team members.

%%% PRACE GENERIC LAYOUT; DO NOT CHANGE %%%
\end{document}
%%% END OF PRACE GENERIC LAYOUT %%%
